\documentclass{tufte-handout}

\title{Discrete Probabilistic Programming Languages\thanks{CS7470 Fall 2023: Foundations of Probabilistic Programming.}}


\newcommand{\varset}[0]{\mathcal{V}}

\author[]{Steven Holtzen\\s.holtzen@northeastern.edu}

%\date{28 March 2010} % without \date command, current date is supplied

%\geometry{showframe} % display margins for debugging page layout
\setcounter{secnumdepth}{1}

\usepackage{graphicx} % allow embedded images
  \setkeys{Gin}{width=\linewidth,totalheight=\textheight,keepaspectratio}
  \graphicspath{{graphics/}} % set of paths to search for images
\usepackage{amsmath,amssymb,amsthm}  % extended mathematics
\usepackage{booktabs} % book-quality tables
\usepackage{units}    % non-stacked fractions and better unit spacing
\usepackage{multicol} % multiple column layout facilities
\usepackage{lipsum}   % filler text
\usepackage{fancyvrb} % extended verbatim environments
  \fvset{fontsize=\normalsize}% default font size for fancy-verbatim environments
\usepackage{listings}
\usepackage{tikz}
\usepackage{mathpartir}
\usepackage{mdframed}
\usepackage{epigraph}
\usepackage{enumitem}
\usepackage{stmaryrd}

\usepackage[ruled,linesnumbered]{algorithm2e}
\SetKwComment{Comment}{/* }{ */}
\newcommand{\indep}{\perp \!\!\! \perp}

\tikzset{
  treenode/.style = {shape=rectangle, rounded corners,
                     draw, align=center,
                     },
  root/.style     = {treenode, font=\Large, bottom color=red!30},
  env/.style      = {treenode, font=\ttfamily\normalsize},
  dummy/.style    = {circle,draw}
}



\usetikzlibrary{positioning}

\newtheorem{theorem}{Theorem}
\newtheorem{definition}{Definition}
\newtheorem{lemma}{Lemma}
\newtheorem{exercise}{Exercise}
\newtheorem{remark}{Remark}


\usepackage{xcolor}

\definecolor{codegreen}{rgb}{0,0.6,0}
\definecolor{codegray}{rgb}{0.5,0.5,0.5}
\definecolor{codepurple}{rgb}{0.58,0,0.82}
\definecolor{backcolour}{rgb}{0.95,0.95,0.92}

\lstdefinestyle{mystyle}{
    backgroundcolor=\color{backcolour},   
    commentstyle=\color{codegreen},
    keywordstyle=\color{magenta},
    numberstyle=\tiny\color{codegray},
    stringstyle=\color{codepurple},
    basicstyle=\ttfamily\footnotesize,
    breakatwhitespace=false,         
    breaklines=true,                 
    captionpos=b,                    
    keepspaces=true,                 
    numbers=left,                    
    numbersep=5pt,                  
    showspaces=false,                
    showstringspaces=false,
    showtabs=false,                  
    tabsize=2
}

\lstset{style=mystyle}

\newcommand{\defn}[1]{\textbf{#1}}
\newcommand{\dbracket}[1]{\llbracket {#1} \rrbracket}
\newcommand{\true}[0]{\texttt{true}}
\newcommand{\false}[0]{\texttt{false}}
\newcommand{\real}[0]{\mathbb{R}}
\newcommand{\rational}[0]{\mathbb{Q}}
\newcommand{\bool}[0]{\mathbb{B}}
\newcommand{\prop}[0]{\textsc{Prop}}


% Standardize command font styles and environments
\newcommand{\doccmd}[1]{\texttt{\textbackslash#1}}% command name -- adds backslash automatically
\newcommand{\docopt}[1]{\ensuremath{\langle}\textrm{\textit{#1}}\ensuremath{\rangle}}% optional command argument
\newcommand{\docarg}[1]{\textrm{\textit{#1}}}% (required) command argument
\newcommand{\docenv}[1]{\textsf{#1}}% environment name
\newcommand{\docpkg}[1]{\texttt{#1}}% package name
\newcommand{\doccls}[1]{\texttt{#1}}% document class name
\newcommand{\docclsopt}[1]{\texttt{#1}}% document class option name
\newenvironment{docspec}{\begin{quote}\noindent}{\end{quote}}% command specification environment



\begin{document}
\maketitle% this prints the handout title, author, and date

\begin{itemize}
  \item Goal for today: compile a simple discrete PPL to BDD
\end{itemize}

\section{Compositional compilation of BDDs}
\begin{itemize}
  \item So far we have compiled by inducting on variables
  \item \textbf{Problem}: this is not \emph{compositional}! A compilation
  process is compositional if it works by compiling big programs out of 
  smaller sub-programs. I.e., it would have a rule that looks something like:
  \begin{mathpar}
    \inferrule{\alpha \compiles \alpha' \and \beta \compiles \beta'}{
      \alpha \land \beta \compiles \alpha' \times \beta'
    }
  \end{mathpar}
  \item Compositional compilation is great: gives us modular reasoning 
  about performance, ... (other reasons?)
  \item \textbf{Goal}: Design a compositional process for compiling \prop$_S$ to 
  \bdd{}

  \item How can we build big BDDs out of smaller ones? Define a way to compose together 
  BDDs, again using a type-directed step-relation $\Gamma \vdash \alpha_1 \land
  \alpha_2 \Downarrow \beta$, shown in Figure~\ref{fig:bdd}.

  
  \begin{theorem}[Correctness]
    If $\Gamma \vdash \alpha_1$, $\Gamma \vdash \alpha_2$, and $\Gamma \vdash \alpha_1 \land \alpha_2 \Downarrow \beta$, 
    then $\dbracket{\alpha_1} \cap \dbracket{\alpha_2} = \dbracket{\beta}$.
  \end{theorem}
  \begin{proof}
    By simultaneous structural induction on syntax of BDDs (note that we need to
    perform simultaneous induction since there are two BDDs at play here). The
    rules in Figure~\ref{fig:bdd} are exhaustive (i.e., every pair of syntactic BDDs matches
    exactly one of these structural rules), so we can proceed by case analysis on each
    of the compilation rules. The base cases are quite simple and we elide them here. The
    interesting cases are the inductive cases.

    We will show the case for \textsc{(SameVarNE)}.  Assume that $\Gamma \vdash \alpha_1 \land
    \alpha_3 \Downarrow \alpha_{13}$ and $\Gamma \vdash \alpha_2 \land \alpha_4 \Downarrow
    \alpha_{24}$. As usual, our inductive hypothesis assumes that the theorem 
    holds for compiled subterms:
    \begin{itemize}
      \item $\dbracket{\Gamma \vdash \alpha_1} \cap \dbracket{\Gamma \vdash \alpha_3} = \dbracket{\Gamma \vdash \alpha_{13}}$
      \item $\dbracket{\Gamma \vdash \alpha_2} \cap \dbracket{\Gamma \vdash \alpha_4} = \dbracket{\Gamma \vdash \alpha_{24}}$
    \end{itemize}

    We want to show that:

    \begin{align*}
    \dbracket{x :: \Gamma \vdash \begin{tikzpicture}
      \node (x) [bddnode] {$x$};
      \node (a1) at ($(x) + (-25bp, -20pt)$) [bddtriangle] {$\alpha_1$};
      \node (a2) at ($(x) + (25bp, -20pt)$) [bddtriangle] {$\alpha_2$};
    \begin{scope}[on background layer]
      \draw [highedge] (x) -- (a1);
      \draw [lowedge] (x) -- (a2);
    \end{scope}
    \end{tikzpicture}}
    \bigcap
    \dbracket{x :: \Gamma \vdash
    \begin{tikzpicture}
      \node (x) [bddnode] {$x$};
      \node (a1) at ($(x) + (-25bp, -20pt)$) [bddtriangle] {$\alpha_3$};
      \node (a2) at ($(x) + (25bp, -20pt)$) [bddtriangle] {$\alpha_4$};
    \begin{scope}[on background layer]
      \draw [highedge] (x) -- (a1);
      \draw [lowedge] (x) -- (a2);
    \end{scope}
    \end{tikzpicture}
    } =
    \dbracket{x :: \Gamma \vdash 
    \begin{tikzpicture}
      \node (x) [bddnode] {$x$};
      \node (a1) at ($(x) + (-25bp, -20pt)$) [bddtriangle] {$\alpha_{13}$};
      \node (a2) at ($(x) + (25bp, -20pt)$) [bddtriangle] {$\alpha_{24}$};
    \begin{scope}[on background layer]
      \draw [highedge] (x) -- (a1);
      \draw [lowedge] (x) -- (a2);
    \end{scope}
    \end{tikzpicture}
    }
  \end{align*}

  We reason forward:
  {\footnotesize 
  \begin{align*}
    \dbracket{x :: \Gamma \vdash 
    \begin{tikzpicture}
      \node (x) [bddnode] {$x$};
      \node (a1) at ($(x) + (-25bp, -20pt)$) [bddtriangle] {$\alpha_{13}$};
      \node (a2) at ($(x) + (25bp, -20pt)$) [bddtriangle] {$\alpha_{24}$};
    \begin{scope}[on background layer]
      \draw [highedge] (x) -- (a1);
      \draw [lowedge] (x) -- (a2);
    \end{scope}
    \end{tikzpicture}
    } &=
    [x \mapsto \true] \otimes \dbracket{\Gamma \vdash \alpha_{13}} ~\bigcup~ [x \mapsto \false] \otimes \dbracket{\Gamma \vdash \alpha_{24}} \\
    &= [x \mapsto \true] \otimes \Big( \dbracket{\Gamma \vdash \alpha_1} \cap \dbracket{\Gamma \vdash \alpha_3} \Big)
      ~\bigcup~ [x \mapsto \false] \otimes \Big( \dbracket{\Gamma \vdash \alpha_2} \cap \dbracket{\Gamma \vdash \alpha_4} \Big)
      & \text{By I.H.} \\
    &= \Big( [x \mapsto \true] \dbracket{\Gamma \vdash \alpha_1} \cap [x \mapsto \true] \dbracket{\Gamma \vdash \alpha_3} \Big)
      ~\bigcup~  \Big([x \mapsto \false] \dbracket{\Gamma \vdash \alpha_2} \cap [x \mapsto \false]\dbracket{\Gamma \vdash \alpha_4} \Big)
      & (\star)\\
    &= \Big( [x \mapsto \true] \dbracket{\Gamma \vdash \alpha_1} \cup [x \mapsto \false] \dbracket{\Gamma \vdash \alpha_2} \Big)
      ~\bigcap~  \Big([x \mapsto \true] \dbracket{\Gamma \vdash \alpha_1} \cup [x \mapsto \false]\dbracket{\Gamma \vdash \alpha_4} \Big)
      & (\dagger) \\ 
    &=     \dbracket{x :: \Gamma \vdash \begin{tikzpicture}
      \node (x) [bddnode] {$x$};
      \node (a1) at ($(x) + (-25bp, -20pt)$) [bddtriangle] {$\alpha_1$};
      \node (a2) at ($(x) + (25bp, -20pt)$) [bddtriangle] {$\alpha_2$};
    \begin{scope}[on background layer]
      \draw [highedge] (x) -- (a1);
      \draw [lowedge] (x) -- (a2);
    \end{scope}
    \end{tikzpicture}}
    \bigcap
    \dbracket{x :: \Gamma \vdash
    \begin{tikzpicture}
      \node (x) [bddnode] {$x$};
      \node (a1) at ($(x) + (-25bp, -20pt)$) [bddtriangle] {$\alpha_3$};
      \node (a2) at ($(x) + (25bp, -20pt)$) [bddtriangle] {$\alpha_4$};
    \begin{scope}[on background layer]
      \draw [highedge] (x) -- (a1);
      \draw [lowedge] (x) -- (a2);
    \end{scope}
    \end{tikzpicture}
    }
  \end{align*}
  }

  where $(\star)$ follows from a simple lemma that $\otimes$ distributes over intersection 
  and $(\dagger)$ follows from distributivity properties of union and intersection, 
  in particular the fact that for any sets $A,B,C,D$ it is the case that $(A \cup B) \cap (C \cup D) = 
  (A \cap C) \cup (B \cap D)$.\sidenote{This set theory property has a nice
  ``proof by Venn-diagram''; draw the Venn-diagram of these sets to see this
  fact clearly.}
  
  \end{proof}

  \item We would also like to ensure that our compilation rules produce reduced and ordered 
  BDDs. We can formalize this with a type-preservation theorem:
  \begin{theorem}[Type preservation]
    If $\Gamma \vdash \alpha_1$, $\Gamma \vdash \alpha_2$, and $\Gamma \vdash \alpha_1 \land \alpha_2 \Downarrow \alpha$, 
    then $\Gamma \vdash \alpha$.
  \end{theorem}

  \item These rules are called \emph{bottom-up BDD compilation}~\citep{darwiche2002knowledge,oztok2015top}. 
  \item There are several other compositional BDD operations we won't have time
  to explain in lecture, but you can use in your lab, like disjunction, negation, substitution, 
  and existential quantification.
  \item Why might one prefer one mode of compilation over the other? Do they
  have different runtime cost? \emph{Yes!}
  \item \textbf{Exercise}: Give example families of formulae where
  top-down compilation is faster than bottom-up and vice-versa.
\end{itemize}

\section{\disc{}: A simple discrete PPL}

\begin{itemize}
  \item Syntax:
\begin{lstlisting}[mathescape=true]
  e ::=
  | x $\leftarrow$ e; e
  | observe e; e
  | flip q                         // q is a rational value
  | if e then e else e
  | return e
  | true | false
  | e $\land$ e | e $\lor$ e | $\neg$ e |
  | ( e )
p ::= e
\end{lstlisting}

\item \disc{} looks very similar to a standard functional programming language,
but has two some interesting new keywords: \texttt{flip}, \texttt{observe}, and
\texttt{return}

\item $\texttt{flip}~\theta$ allocates a new random quantity that is $\true$
with probability $\theta$ and $\false$ with probability $1-\theta$

\item \texttt{return e} turns a non-probabilistic quantity into a probabilistic one, i.e. 
\texttt{return true} is a \emph{probabilistic quantity} that is $\true$ with probability 1 and 
$\false$ with probability 0

\item Example program and its interpretation: 

\begin{lstlisting}[mathescape=true]
x $\leftarrow$ flip 0.5; 
y $\leftarrow$ flip 0.5;
return x $\land$ y
\end{lstlisting}

This program outputs the probability distribution $[\true \mapsto 0.25, \false
\mapsto 0.75]$.

\item \texttt{observe} is a powerful keyword that lets us \emph{condition} the
program. For instance, suppose I want to model the following scenario: ``flip 
two coins and observe that at least one of them is heads. What is the probability 
that the first coin was heads?''

We can encode this scenario as a \disc{} program:

\begin{lstlisting}[mathescape=true]
x $\leftarrow$ flip 0.5; 
y $\leftarrow$ flip 0.5;
observe x $\lor$ y;
return x
\end{lstlisting}

This program outputs the probability distribution:
\begin{align*}
  [\true \mapsto (0.25 + 0.25) / 0.75, \false
\mapsto 0.25 / 0.75]
\end{align*} 

\item \textbf{Type system}: terms can either be pure Booleans of type $\bool$ 
or distributions on Booleans of type $\dist{\bool}$. So, we have the following 
type definition:
\begin{align}
  \tau ::= \bool \mid \dist{\bool}.
\end{align}

\item We define a typing judgment $\Gamma \vdash \te : \tau$ that associates each 
term with a type. The typing context $\Gamma$ is a map from identifiers to types.
\begin{mathpar}
  \inferrule{}{\Gamma \vdash \true{} : \bool} \and 
  \inferrule{}{\Gamma \vdash \false{} : \bool} \and
  \inferrule{}{\Gamma \vdash \texttt{flip}~\theta : \dist{\bool}} \and 
  \inferrule{\Gamma \vdash \te : \bool}{\Gamma \vdash \texttt{return}~\te : \dist{\bool}} \and
  \inferrule{\Gamma \vdash \te_1 : \bool \and \Gamma \vdash \te_2 : \tau}
    {\Gamma \vdash \texttt{observe}~\te_1; \te_2 : \tau} \and
  \inferrule{\Gamma \vdash \te_1 : \dist{\bool} \and \Gamma \cup [x \mapsto \bool] \vdash \te_2 : \tau}
    {\Gamma \vdash x \leftarrow \te_1; \te_2 : \tau} \and 
  \inferrule{\Gamma \vdash \te_1 : \bool \and \Gamma \vdash \te_2 : \tau \and \Gamma \vdash \te_3 : \tau}
    {\Gamma \vdash \texttt{if}~\te_1~\texttt{then}~\te_2~\texttt{else}~\te_3 : \tau} \and
  \inferrule{\Gamma \vdash \te_1 : \bool \and \Gamma \vdash \te_2 : \bool}
    {\Gamma \vdash \te_1 \land \te_2 : \bool}
\end{mathpar}

\end{itemize}



\subsection{Denotational semantics of \disc{}}

\begin{itemize}
\item Associates each term with an \emph{unnormalized probability distribution} 
(i.e., the total probability mass may be less than 1).
\item Has the type $\dbracket{\te} : \texttt{Bool} \rightarrow [0, 1]$,
and has the following inductive definition:

$$
[\![\texttt{flip}~\theta]\!](v) = 
\begin{cases}
\theta& \quad \text{if }v = T\\
1-\theta& \quad \text{if }v = F\\
\end{cases}
$$

$$
 [\![\texttt{return}~e]\!](v) = 
 \begin{cases}
 1\quad& \text{if }v = [\![e]\!]\\
 0\quad&  \text{otherwise}\\
 \end{cases}
$$

$$
[\![x \leftarrow e_1; e_2]\!](v) = \sum_{v'} [\![{e_1}]\!](v') \times [\![{e_2[x \mapsto v']}]\!](v)
$$

$$
[\![\texttt{observe}~e_1; e_2]\!](v) = 
\sum_{\{v' \mid [\![{e_1}(v') = T\}]\!]} [\![{e_2}]\!](v)
$$

\item The semantics for non-probabilistic terms is standard. The semantic evaluation has type 
$[[e]]: \texttt{Bool}$ for closed terms.

\item These semantics give an unnormalized distribution. The main semantic object of interest is 
the normalized distribution, which is given by the \defn{normalized semantics}:

$$
[\![e]\!]_D(T) = \frac{[\![e]\!](T)}{[\![e]\!](T) + [\![e]\!](F) },
$$

defined analogously for the false case.
\end{itemize}

% \subsection{Inference via enumeration}
% \begin{itemize}
%   \item Define a relation $\te \Downarrow^e $
% \end{itemize}

\subsection{Compiling \disc{} to \prop{}}
\begin{itemize}
  \item \textbf{Goal}: give a semantics-preserving compilation
  $\rightsquigarrow$ that compiles \disc{} to \prop{}
  \item In order to handle observations, we will compile \disc{} programs into 
  \emph{two} \prop{} programs: one that computes the unnormalized probability of 
  returning $\true$, and one that computes the probability of evidence (i.e. normalizing constant)
  \item Inductive description has the shape $\te \compiles (\texttt{p}_1, \texttt{p}_2)$. We want to 
  define this relation to satisfy the following \textbf{adequacy condition}:
  \begin{align} 
    \dbracket{\te}_D(\true) = \frac{\dbracket{\prog_1}}{\dbracket{\prog_2}}.
  \end{align} 
  We will shorten this description to $\te \compiles
  (\varphi, \varphi_A, w)$ and assume that the two formulae share a common
  $w$. The adequacy condition then becomes:
  \begin{align} 
    \dbracket{\te}_D(\true) = \frac{\dbracket{(\varphi, w)}}{\dbracket{(\varphi_A, w)}}.
  \end{align} 
  \item Compilation relation:
  \begin{mathpar}
    \inferrule{}{\true \compiles (\true, \true, \emptyset)} \and 
    \inferrule{}{\false \compiles (\false, \true, \emptyset)} \and 
    \inferrule{}{x \compiles (x, \true, \emptyset)} \and 
    \inferrule{\texttt{fresh}~x}
      {\texttt{flip}~\theta \compiles (x, \true, [x \mapsto \theta, \overline{x} \mapsto 1-\theta])} \and
    \inferrule{\te_1 \compiles (\varphi, \varphi_A, w) \and \te_2 \compiles (\varphi', \varphi_A', w')}
      {x \leftarrow \te_1; \te_2 \compiles (\varphi'[\varphi/x], \varphi_A'[\varphi/x] \land \varphi_A, w_1 \cup w_2)} \and
    \inferrule{\te_1 \compiles (\varphi, \varphi_A, w) \and \te_2 \compiles (\varphi', \varphi_A', w')}
      {\texttt{observe}~\te_1; \te_2 \compiles (\varphi', \varphi_A' \land \varphi_A, w_1 \cup w_2)}
  \end{mathpar}

\end{itemize}
\begin{theorem}[Adequacy]
  For well-typed term $\te$,
  assume $\te \compiles (\varphi, \varphi_A, w)$. Then, 
  $\dbracket{\te}_D(\true) = {\dbracket{(\varphi, w)}} / {\dbracket{(\varphi_A, w)}}$.
\end{theorem}

  \begin{figure}
  \begin{mathpar}
    \inferrule{}{\Gamma \vdash \bddtrue{} \land \alpha \Downarrow \alpha} \and 
    \inferrule{}{\Gamma \vdash \alpha \land \bddtrue{} \Downarrow \alpha} \\
    \inferrule{}{\Gamma \vdash \bddfalse{} \land \alpha \Downarrow \bddfalse{}} \and 
    \inferrule{}{\Gamma \vdash \alpha \land \bddfalse{} \Downarrow \bddfalse{}} \\

    \inferrule*[Right={\textsc{(SameVarNE)}}]{\Gamma \vdash \alpha_1 \land \alpha_3 \compiles \alpha_{13} \and 
    \Gamma \vdash \alpha_2 \land \alpha_4 \compiles \alpha_{24} \and 
    \alpha_{13} \ne \alpha_{24}
    }{x :: \Gamma \vdash 
    \begin{tikzpicture}
      \node (x) [bddnode] {$x$};
      \node (a1) at ($(x) + (-25bp, -20pt)$) [bddtriangle] {$\alpha_1$};
      \node (a2) at ($(x) + (25bp, -20pt)$) [bddtriangle] {$\alpha_2$};
    \begin{scope}[on background layer]
      \draw [highedge] (x) -- (a1);
      \draw [lowedge] (x) -- (a2);
    \end{scope}
    \end{tikzpicture}
    \land
    \begin{tikzpicture}
      \node (x) [bddnode] {$x$};
      \node (a1) at ($(x) + (-25bp, -20pt)$) [bddtriangle] {$\alpha_3$};
      \node (a2) at ($(x) + (25bp, -20pt)$) [bddtriangle] {$\alpha_4$};
    \begin{scope}[on background layer]
      \draw [highedge] (x) -- (a1);
      \draw [lowedge] (x) -- (a2);
    \end{scope}
    \end{tikzpicture}
    \Downarrow 
    \begin{tikzpicture}
      \node (x) [bddnode] {$x$};
      \node (a1) at ($(x) + (-25bp, -20pt)$) [bddtriangle] {$\alpha_{13}$};
      \node (a2) at ($(x) + (25bp, -20pt)$) [bddtriangle] {$\alpha_{24}$};
    \begin{scope}[on background layer]
      \draw [highedge] (x) -- (a1);
      \draw [lowedge] (x) -- (a2);
    \end{scope}
    \end{tikzpicture}
    }

    \\

    \inferrule*[Right={\textsc{(SameVarEQ)}}]{\Gamma \vdash \alpha_1 \land \alpha_3 \compiles \alpha_{13} \and 
    \Gamma \vdash \alpha_2 \land \alpha_4 \compiles \alpha_{24} \and 
    \alpha_{13} = \alpha_{24}
    }{x :: \Gamma \vdash 
    \begin{tikzpicture}
      \node (x) [bddnode] {$x$};
      \node (a1) at ($(x) + (-25bp, -20pt)$) [bddtriangle] {$\alpha_1$};
      \node (a2) at ($(x) + (25bp, -20pt)$) [bddtriangle] {$\alpha_2$};
    \begin{scope}[on background layer]
      \draw [highedge] (x) -- (a1);
      \draw [lowedge] (x) -- (a2);
    \end{scope}
    \end{tikzpicture}
    \land
    \begin{tikzpicture}
      \node (x) [bddnode] {$x$};
      \node (a1) at ($(x) + (-25bp, -20pt)$) [bddtriangle] {$\alpha_3$};
      \node (a2) at ($(x) + (25bp, -20pt)$) [bddtriangle] {$\alpha_4$};
    \begin{scope}[on background layer]
      \draw [highedge] (x) -- (a1);
      \draw [lowedge] (x) -- (a2);
    \end{scope}
    \end{tikzpicture}
    \Downarrow 
    \bddtriangle{$\alpha_{24}$}
    } 

    \\

    \inferrule*[Right=\textsc{(Weaken)}]{x \ne y \ne z \and 
    \Gamma \vdash 
    \begin{tikzpicture}
      \node (x) [bddnode] {$y$};
      \node (a1) at ($(x) + (-25bp, -20pt)$) [bddtriangle] {$\alpha_1$};
      \node (a2) at ($(x) + (25bp, -20pt)$) [bddtriangle] {$\alpha_2$};
    \begin{scope}[on background layer]
      \draw [highedge] (x) -- (a1);
      \draw [lowedge] (x) -- (a2);
    \end{scope}
    \end{tikzpicture}
    \land
    \begin{tikzpicture}
      \node (x) [bddnode] {$z$};
      \node (a1) at ($(x) + (-25bp, -20pt)$) [bddtriangle] {$\alpha_3$};
      \node (a2) at ($(x) + (25bp, -20pt)$) [bddtriangle] {$\alpha_4$};
    \begin{scope}[on background layer]
      \draw [highedge] (x) -- (a1);
      \draw [lowedge] (x) -- (a2);
    \end{scope}
    \end{tikzpicture}
    \Downarrow 
    \bddtriangle{$\alpha$}
    }{x :: \Gamma \vdash 
    \begin{tikzpicture}
      \node (x) [bddnode] {$y$};
      \node (a1) at ($(x) + (-25bp, -20pt)$) [bddtriangle] {$\alpha_1$};
      \node (a2) at ($(x) + (25bp, -20pt)$) [bddtriangle] {$\alpha_2$};
    \begin{scope}[on background layer]
      \draw [highedge] (x) -- (a1);
      \draw [lowedge] (x) -- (a2);
    \end{scope}
    \end{tikzpicture}
    \land
    \begin{tikzpicture}
      \node (x) [bddnode] {$z$};
      \node (a1) at ($(x) + (-25bp, -20pt)$) [bddtriangle] {$\alpha_3$};
      \node (a2) at ($(x) + (25bp, -20pt)$) [bddtriangle] {$\alpha_4$};
    \begin{scope}[on background layer]
      \draw [highedge] (x) -- (a1);
      \draw [lowedge] (x) -- (a2);
    \end{scope}
    \end{tikzpicture}
    \Downarrow 
    \bddtriangle{$\alpha$}
    }

    \\

    \inferrule*[Right=\textsc{(ParNE)}]{x \ne y \and 
    \Gamma \vdash 
    \begin{tikzpicture}
      \node (x) [bddnode] {$y$};
      \node (a1) at ($(x) + (-25bp, -20pt)$) [bddtriangle] {$\alpha_1$};
      \node (a2) at ($(x) + (25bp, -20pt)$) [bddtriangle] {$\alpha_2$};
    \begin{scope}[on background layer]
      \draw [highedge] (x) -- (a1);
      \draw [lowedge] (x) -- (a2);
    \end{scope}
    \end{tikzpicture}
    \land
    \bddtriangle{$\alpha_1$}
    \Downarrow 
    \alpha_{y_1}
    \and 
    \Gamma \vdash 
    \begin{tikzpicture}
      \node (x) [bddnode] {$y$};
      \node (a1) at ($(x) + (-25bp, -20pt)$) [bddtriangle] {$\alpha_1$};
      \node (a2) at ($(x) + (25bp, -20pt)$) [bddtriangle] {$\alpha_2$};
    \begin{scope}[on background layer]
      \draw [highedge] (x) -- (a1);
      \draw [lowedge] (x) -- (a2);
    \end{scope}
    \end{tikzpicture}
    \land
    \bddtriangle{$\alpha_2$}
    \Downarrow 
    \alpha_{y_2}
    \and 
    \alpha_{y_1} \ne \alpha_{y_2}
    }{x :: \Gamma \vdash 
    \begin{tikzpicture}
      \node (x) [bddnode] {$x$};
      \node (a1) at ($(x) + (-25bp, -20pt)$) [bddtriangle] {$\alpha_1$};
      \node (a2) at ($(x) + (25bp, -20pt)$) [bddtriangle] {$\alpha_2$};
    \begin{scope}[on background layer]
      \draw [highedge] (x) -- (a1);
      \draw [lowedge] (x) -- (a2);
    \end{scope}
    \end{tikzpicture}
    \land
    \begin{tikzpicture}
      \node (x) [bddnode] {$y$};
      \node (a1) at ($(x) + (-25bp, -20pt)$) [bddtriangle] {$\alpha_3$};
      \node (a2) at ($(x) + (25bp, -20pt)$) [bddtriangle] {$\alpha_4$};
    \begin{scope}[on background layer]
      \draw [highedge] (x) -- (a1);
      \draw [lowedge] (x) -- (a2);
    \end{scope}
    \end{tikzpicture}
    \Downarrow 
    \begin{tikzpicture}
      \node (x) [bddnode] {$x$};
      \node (a1) at ($(x) + (-25bp, -20pt)$) [bddtriangle] {$\alpha_{y_1}$};
      \node (a2) at ($(x) + (25bp, -20pt)$) [bddtriangle] {$\alpha_{y_2}$};
    \begin{scope}[on background layer]
      \draw [highedge] (x) -- (a1);
      \draw [lowedge] (x) -- (a2);
    \end{scope}
    \end{tikzpicture}
    }

    \\

    \inferrule*[Right=\textsc{(ParEQ)}]{x \ne y \and 
    \Gamma \vdash 
    \begin{tikzpicture}
      \node (x) [bddnode] {$y$};
      \node (a1) at ($(x) + (-25bp, -20pt)$) [bddtriangle] {$\alpha_1$};
      \node (a2) at ($(x) + (25bp, -20pt)$) [bddtriangle] {$\alpha_2$};
    \begin{scope}[on background layer]
      \draw [highedge] (x) -- (a1);
      \draw [lowedge] (x) -- (a2);
    \end{scope}
    \end{tikzpicture}
    \land
    \bddtriangle{$\alpha_1$}
    \Downarrow 
    \alpha_{y_1}
    \and 
    \Gamma \vdash 
    \begin{tikzpicture}
      \node (x) [bddnode] {$y$};
      \node (a1) at ($(x) + (-25bp, -20pt)$) [bddtriangle] {$\alpha_1$};
      \node (a2) at ($(x) + (25bp, -20pt)$) [bddtriangle] {$\alpha_2$};
    \begin{scope}[on background layer]
      \draw [highedge] (x) -- (a1);
      \draw [lowedge] (x) -- (a2);
    \end{scope}
    \end{tikzpicture}
    \land
    \bddtriangle{$\alpha_2$}
    \Downarrow 
    \alpha_{y_2}
    \and 
    \alpha_{y_1} = \alpha_{y_2}
    }{x :: \Gamma \vdash 
    \begin{tikzpicture}
      \node (x) [bddnode] {$x$};
      \node (a1) at ($(x) + (-25bp, -20pt)$) [bddtriangle] {$\alpha_1$};
      \node (a2) at ($(x) + (25bp, -20pt)$) [bddtriangle] {$\alpha_2$};
    \begin{scope}[on background layer]
      \draw [highedge] (x) -- (a1);
      \draw [lowedge] (x) -- (a2);
    \end{scope}
    \end{tikzpicture}
    \land
    \begin{tikzpicture}
      \node (x) [bddnode] {$y$};
      \node (a1) at ($(x) + (-25bp, -20pt)$) [bddtriangle] {$\alpha_3$};
      \node (a2) at ($(x) + (25bp, -20pt)$) [bddtriangle] {$\alpha_4$};
    \begin{scope}[on background layer]
      \draw [highedge] (x) -- (a1);
      \draw [lowedge] (x) -- (a2);
    \end{scope}
    \end{tikzpicture}
    \Downarrow 
    \alpha_{y_2}
    }
  \end{mathpar}
  \caption{Rules for compiling BDDs.}
  \label{fig:bdd}
\end{figure}

\bibliographystyle{plainnat}
\bibliography{../bib}


\end{document}