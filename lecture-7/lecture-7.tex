\documentclass{tufte-handout}

\title{Discrete Probabilistic Programming Languages\thanks{CS7470 Fall 2023: Foundations of Probabilistic Programming.}}


\newcommand{\varset}[0]{\mathcal{V}}

\author[]{Steven Holtzen\\s.holtzen@northeastern.edu}

%\date{28 March 2010} % without \date command, current date is supplied

%\geometry{showframe} % display margins for debugging page layout
\setcounter{secnumdepth}{1}

\usepackage{graphicx} % allow embedded images
  \setkeys{Gin}{width=\linewidth,totalheight=\textheight,keepaspectratio}
  \graphicspath{{graphics/}} % set of paths to search for images
\usepackage{amsmath,amssymb,amsthm}  % extended mathematics
\usepackage{booktabs} % book-quality tables
\usepackage{units}    % non-stacked fractions and better unit spacing
\usepackage{multicol} % multiple column layout facilities
\usepackage{lipsum}   % filler text
\usepackage{fancyvrb} % extended verbatim environments
  \fvset{fontsize=\normalsize}% default font size for fancy-verbatim environments
\usepackage{listings}
\usepackage{tikz}
\usepackage{mathpartir}
\usepackage{mdframed}
\usepackage{epigraph}
\usepackage{enumitem}
\usepackage{stmaryrd}

\usepackage[ruled,linesnumbered]{algorithm2e}
\SetKwComment{Comment}{/* }{ */}
\newcommand{\indep}{\perp \!\!\! \perp}

\tikzset{
  treenode/.style = {shape=rectangle, rounded corners,
                     draw, align=center,
                     },
  root/.style     = {treenode, font=\Large, bottom color=red!30},
  env/.style      = {treenode, font=\ttfamily\normalsize},
  dummy/.style    = {circle,draw}
}



\usetikzlibrary{positioning}

\newtheorem{theorem}{Theorem}
\newtheorem{definition}{Definition}
\newtheorem{lemma}{Lemma}
\newtheorem{exercise}{Exercise}
\newtheorem{remark}{Remark}


\usepackage{xcolor}

\definecolor{codegreen}{rgb}{0,0.6,0}
\definecolor{codegray}{rgb}{0.5,0.5,0.5}
\definecolor{codepurple}{rgb}{0.58,0,0.82}
\definecolor{backcolour}{rgb}{0.95,0.95,0.92}

\lstdefinestyle{mystyle}{
    backgroundcolor=\color{backcolour},   
    commentstyle=\color{codegreen},
    keywordstyle=\color{magenta},
    numberstyle=\tiny\color{codegray},
    stringstyle=\color{codepurple},
    basicstyle=\ttfamily\footnotesize,
    breakatwhitespace=false,         
    breaklines=true,                 
    captionpos=b,                    
    keepspaces=true,                 
    numbers=left,                    
    numbersep=5pt,                  
    showspaces=false,                
    showstringspaces=false,
    showtabs=false,                  
    tabsize=2
}

\lstset{style=mystyle}

\newcommand{\defn}[1]{\textbf{#1}}
\newcommand{\dbracket}[1]{\llbracket {#1} \rrbracket}
\newcommand{\true}[0]{\texttt{true}}
\newcommand{\false}[0]{\texttt{false}}
\newcommand{\real}[0]{\mathbb{R}}
\newcommand{\rational}[0]{\mathbb{Q}}
\newcommand{\bool}[0]{\mathbb{B}}
\newcommand{\prop}[0]{\textsc{Prop}}


% Standardize command font styles and environments
\newcommand{\doccmd}[1]{\texttt{\textbackslash#1}}% command name -- adds backslash automatically
\newcommand{\docopt}[1]{\ensuremath{\langle}\textrm{\textit{#1}}\ensuremath{\rangle}}% optional command argument
\newcommand{\docarg}[1]{\textrm{\textit{#1}}}% (required) command argument
\newcommand{\docenv}[1]{\textsf{#1}}% environment name
\newcommand{\docpkg}[1]{\texttt{#1}}% package name
\newcommand{\doccls}[1]{\texttt{#1}}% document class name
\newcommand{\docclsopt}[1]{\texttt{#1}}% document class option name
\newenvironment{docspec}{\begin{quote}\noindent}{\end{quote}}% command specification environment



\begin{document}
\maketitle% this prints the handout title, author, and date

\section{\disc{}: A simple discrete PPL}

\begin{itemize}
  \item Syntax:
\begin{lstlisting}[mathescape=true]
  e ::=
  | x $\leftarrow$ e; e
  | observe e; e
  | flip q                         // q is a rational value
  | if e then e else e
  | return e
  | true | false
  | e $\land$ e | e $\lor$ e | $\neg$ e |
  | ( e )
p ::= e
\end{lstlisting}

\item \disc{} looks very similar to a standard functional programming language,
but has two some interesting new keywords: \texttt{flip}, \texttt{observe}, and
\texttt{return}

\item $\texttt{flip}~\theta$ allocates a new random quantity that is $\true$
with probability $\theta$ and $\false$ with probability $1-\theta$

\item \texttt{return e} turns a non-probabilistic quantity into a probabilistic one, i.e. 
\texttt{return true} is a \emph{probabilistic quantity} that is $\true$ with probability 1 and 
$\false$ with probability 0

\item Example program and its interpretation: 

\begin{lstlisting}[mathescape=true]
x $\leftarrow$ flip 0.5; 
y $\leftarrow$ flip 0.5;
return x $\land$ y
\end{lstlisting}

This program outputs the probability distribution $[\true \mapsto 0.25, \false
\mapsto 0.75]$.

\item \texttt{observe} is a powerful keyword that lets us \emph{condition} the
program. For instance, suppose I want to model the following scenario: ``flip 
two coins and observe that at least one of them is heads. What is the probability 
that the first coin was heads?''

We can encode this scenario as a \disc{} program:

\begin{lstlisting}[mathescape=true]
x $\leftarrow$ flip 0.5; 
y $\leftarrow$ flip 0.5;
observe x $\lor$ y;
return x
\end{lstlisting}

This program outputs the probability distribution:
\begin{align*}
  [\true \mapsto (0.25 + 0.25) / 0.75, \false
\mapsto 0.25 / 0.75]
\end{align*} 

\item \textbf{Type system}: terms can either be pure Booleans of type $\bool$ 
or distributions on Booleans of type $\dist{\bool}$. So, we have the following 
type definition:
\begin{align}
  \tau ::= \bool \mid \dist{\bool}.
\end{align}

\item We define a typing judgment $\Gamma \vdash \te : \tau$ that associates each 
term with a type. The typing context $\Gamma$ is a map from identifiers to types.
\begin{mathpar}
  \inferrule{}{\Gamma \vdash \true{} : \bool} \and 
  \inferrule{}{\Gamma \vdash \false{} : \bool} \and
  \inferrule{}{\Gamma \vdash \texttt{flip}~\theta : \dist{\bool}} \and 
  \inferrule{\Gamma \vdash \te : \bool}{\Gamma \vdash \texttt{return}~\te : \dist{\bool}} \and
  \inferrule{\Gamma \vdash \te_1 : \bool \and \Gamma \vdash \te_2 : \tau}
    {\Gamma \vdash \texttt{observe}~\te_1; \te_2 : \tau} \and
  \inferrule{\Gamma \vdash \te_1 : \dist{\bool} \and \Gamma \cup [x \mapsto \bool] \vdash \te_2 : \tau}
    {\Gamma \vdash x \leftarrow \te_1; \te_2 : \tau} \and 
  \inferrule{\Gamma \vdash \te_1 : \bool \and \Gamma \vdash \te_2 : \tau \and \Gamma \vdash \te_3 : \tau}
    {\Gamma \vdash \texttt{if}~\te_1~\texttt{then}~\te_2~\texttt{else}~\te_3 : \tau} \and
  \inferrule{\Gamma \vdash \te_1 : \bool \and \Gamma \vdash \te_2 : \bool}
    {\Gamma \vdash \te_1 \land \te_2 : \bool}
\end{mathpar}

\end{itemize}



\subsection{Denotational semantics of \disc{}}

\begin{itemize}
\item Associates each term with an \emph{unnormalized probability distribution} 
(i.e., the total probability mass may be less than 1).
\item Has the type $\dbracket{\te} : \texttt{Bool} \rightarrow [0, 1]$,
and has the following inductive definition:

$$
[\![\texttt{flip}~\theta]\!](v) = 
\begin{cases}
\theta& \quad \text{if }v = T\\
1-\theta& \quad \text{if }v = F\\
\end{cases}
$$

$$
 [\![\texttt{return}~e]\!](v) = 
 \begin{cases}
 1\quad& \text{if }v = [\![e]\!]\\
 0\quad&  \text{otherwise}\\
 \end{cases}
$$

$$
[\![x \leftarrow e_1; e_2]\!](v) = \sum_{v'} [\![{e_1}]\!](v') \times [\![{e_2[x \mapsto v']}]\!](v)
$$

$$
[\![\texttt{observe}~e_1; e_2]\!](v) = 
\sum_{\{v' \mid [\![{e_1}(v') = T\}]\!]} [\![{e_2}]\!](v)
$$

\item The semantics for non-probabilistic terms is standard. The semantic evaluation has type 
$[[e]]: \texttt{Bool}$ for closed terms.

\item These semantics give an unnormalized distribution. The main semantic object of interest is 
the normalized distribution, which is given by the \defn{normalized semantics}:

$$
[\![e]\!]_D(T) = \frac{[\![e]\!](T)}{[\![e]\!](T) + [\![e]\!](F) },
$$

defined analogously for the false case.
\end{itemize}

% \subsection{Inference via enumeration}
% \begin{itemize}
%   \item Define a relation $\te \Downarrow^e $
% \end{itemize}

\section{Observation}

\section{Compiling \disc{} to \prop{}}
\begin{itemize}
  \item \textbf{Goal}: give a semantics-preserving compilation
  $\rightsquigarrow$ that compiles \disc{} to \prop{}
  \item In order to handle observations, we will compile \disc{} programs into 
  \emph{two} \prop{} programs: one that computes the unnormalized probability of 
  returning $\true$, and one that computes the probability of evidence (i.e. normalizing constant)
  \item Inductive description has the shape $\te \compiles (\texttt{p}_1, \texttt{p}_2)$. We want to 
  define this relation to satisfy the following \textbf{adequacy condition}:
  \begin{align} 
    \dbracket{\te}_D(\true) = \frac{\dbracket{\prog_1}}{\dbracket{\prog_2}}.
  \end{align} 
  We will shorten this description to $\te \compiles
  (\varphi, \varphi_A, w)$ and assume that the two formulae share a common
  $w$. The adequacy condition then becomes:
  \begin{align} 
    \dbracket{\te}_D(\true) = \frac{\dbracket{(\varphi, w)}}{\dbracket{(\varphi_A, w)}}.
  \end{align} 
  \item Compilation relation:
  \begin{mathpar}
    \inferrule{}{\true \compiles (\true, \true, \emptyset)} \and 
    \inferrule{}{\false \compiles (\false, \true, \emptyset)} \and 
    \inferrule{}{x \compiles (x, \true, \emptyset)} \and 
    \inferrule{\texttt{fresh}~x}
      {\texttt{flip}~\theta \compiles (x, \true, [x \mapsto \theta, \overline{x} \mapsto 1-\theta])} \and
    \inferrule{\te_1 \compiles (\varphi, \varphi_A, w) \and \te_2 \compiles (\varphi', \varphi_A', w')}
      {x \leftarrow \te_1; \te_2 \compiles (\varphi'[\varphi/x], \varphi_A'[\varphi/x] \land \varphi_A, w_1 \cup w_2)} \and
    \inferrule{\te_1 \compiles (\varphi, \varphi_A, w) \and \te_2 \compiles (\varphi', \varphi_A', w')}
      {\texttt{observe}~\te_1; \te_2 \compiles (\varphi', \varphi_A' \land \varphi_A, w_1 \cup w_2)}
  \end{mathpar}

\end{itemize}
\begin{theorem}[Adequacy]
  For well-typed term $\te$,
  assume $\te \compiles (\varphi, \varphi_A, w)$. Then, 
  $\dbracket{\te}_D(\true) = {\dbracket{(\varphi, w)}} / {\dbracket{(\varphi_A, w)}}$.
\end{theorem}

\bibliographystyle{plainnat}
\bibliography{../bib}


\end{document}