\documentclass{tufte-handout}

\title{Discrete Probabilistic Programming Languages III: Observation \& Sampling\thanks{CS7470 Fall 2023: Foundations of Probabilistic Programming.}}


\newcommand{\varset}[0]{\mathcal{V}}

\author[]{Steven Holtzen\\s.holtzen@northeastern.edu}

%\date{28 March 2010} % without \date command, current date is supplied

%\geometry{showframe} % display margins for debugging page layout
\setcounter{secnumdepth}{1}

\usepackage{graphicx} % allow embedded images
  \setkeys{Gin}{width=\linewidth,totalheight=\textheight,keepaspectratio}
  \graphicspath{{graphics/}} % set of paths to search for images
\usepackage{amsmath,amssymb,amsthm}  % extended mathematics
\usepackage{booktabs} % book-quality tables
\usepackage{units}    % non-stacked fractions and better unit spacing
\usepackage{multicol} % multiple column layout facilities
\usepackage{lipsum}   % filler text
\usepackage{fancyvrb} % extended verbatim environments
  \fvset{fontsize=\normalsize}% default font size for fancy-verbatim environments
\usepackage{listings}
\usepackage{tikz}
\usepackage{mathpartir}
\usepackage{mdframed}
\usepackage{epigraph}
\usepackage{enumitem}
\usepackage{stmaryrd}

\usepackage[ruled,linesnumbered]{algorithm2e}
\SetKwComment{Comment}{/* }{ */}
\newcommand{\indep}{\perp \!\!\! \perp}

\tikzset{
  treenode/.style = {shape=rectangle, rounded corners,
                     draw, align=center,
                     },
  root/.style     = {treenode, font=\Large, bottom color=red!30},
  env/.style      = {treenode, font=\ttfamily\normalsize},
  dummy/.style    = {circle,draw}
}



\usetikzlibrary{positioning}

\newtheorem{theorem}{Theorem}
\newtheorem{definition}{Definition}
\newtheorem{lemma}{Lemma}
\newtheorem{exercise}{Exercise}
\newtheorem{remark}{Remark}


\usepackage{xcolor}

\definecolor{codegreen}{rgb}{0,0.6,0}
\definecolor{codegray}{rgb}{0.5,0.5,0.5}
\definecolor{codepurple}{rgb}{0.58,0,0.82}
\definecolor{backcolour}{rgb}{0.95,0.95,0.92}

\lstdefinestyle{mystyle}{
    backgroundcolor=\color{backcolour},   
    commentstyle=\color{codegreen},
    keywordstyle=\color{magenta},
    numberstyle=\tiny\color{codegray},
    stringstyle=\color{codepurple},
    basicstyle=\ttfamily\footnotesize,
    breakatwhitespace=false,         
    breaklines=true,                 
    captionpos=b,                    
    keepspaces=true,                 
    numbers=left,                    
    numbersep=5pt,                  
    showspaces=false,                
    showstringspaces=false,
    showtabs=false,                  
    tabsize=2
}

\lstset{style=mystyle}

\newcommand{\defn}[1]{\textbf{#1}}
\newcommand{\dbracket}[1]{\llbracket {#1} \rrbracket}
\newcommand{\true}[0]{\texttt{true}}
\newcommand{\false}[0]{\texttt{false}}
\newcommand{\real}[0]{\mathbb{R}}
\newcommand{\rational}[0]{\mathbb{Q}}
\newcommand{\bool}[0]{\mathbb{B}}
\newcommand{\prop}[0]{\textsc{Prop}}


% Standardize command font styles and environments
\newcommand{\doccmd}[1]{\texttt{\textbackslash#1}}% command name -- adds backslash automatically
\newcommand{\docopt}[1]{\ensuremath{\langle}\textrm{\textit{#1}}\ensuremath{\rangle}}% optional command argument
\newcommand{\docarg}[1]{\textrm{\textit{#1}}}% (required) command argument
\newcommand{\docenv}[1]{\textsf{#1}}% environment name
\newcommand{\docpkg}[1]{\texttt{#1}}% package name
\newcommand{\doccls}[1]{\texttt{#1}}% document class name
\newcommand{\docclsopt}[1]{\texttt{#1}}% document class option name
\newenvironment{docspec}{\begin{quote}\noindent}{\end{quote}}% command specification environment



\begin{document}
\maketitle% this prints the handout title, author, and date

\section{Observation}
\begin{itemize}
  \item Why observation? 
  \item \texttt{observe} is a powerful keyword that lets us \emph{condition} the
program. For instance, suppose I want to model the following scenario: ``flip 
two coins and observe that at least one of them is heads. What is the probability 
that the first coin was heads?''

We can encode this scenario as a \disc{} program:

\begin{lstlisting}[mathescape=true, caption={\textsc{TwoCoins}}]
x $\leftarrow$ flip 0.5; 
y $\leftarrow$ flip 0.5;
observe x $\lor$ y;
return x
\end{lstlisting}

This program outputs the probability distribution:
\begin{align*}
  [\true \mapsto (0.25 + 0.25) / 0.75, \false
\mapsto 0.25 / 0.75]
\end{align*} 

\item Denotational semantics of \texttt{observe}:
  \begin{align*}
  \dbracket{\texttt{observe}~\te_1; \te_2]}(v) = 
  \sum_{\{v' \mid \dbracket{e_1}(v') = T\}} \dbracket{\te_2}(v)
  \end{align*}

  Note that, with this definition, the semantics of probabilistic terms are now
  \emph{unnormalized} (i.e., the distribution does not sum to 1). For example:
  \begin{align*}
    \dbracket{\false} = \begin{cases}
      \true &\mapsto 0 \\ 
      \false &\mapsto 0
    \end{cases}
  \end{align*}

  The \textsc{TwoCoins} case:
  \begin{align*}
    \dbracket{\textsc{TwoCoins}} = \begin{cases}
      \true &\mapsto 0.5 \\ 
      \false &\mapsto 0.25
    \end{cases}
  \end{align*}
  

\item  The main semantic object of interest is 
the \emph{normalized distribution}, which is given by the \defn{normalized semantics}:
\begin{align*}
\dbracket{\te}_D(T) = \frac{\dbracket{\te}(T)}{\dbracket{\te}(T) + \dbracket{\te}(F) },
\end{align*}


defined analogously for the false case.
  \item In order to handle observations, we will compile \disc{} programs into 
  \emph{two} \prop{} programs: one that computes the unnormalized probability of 
  returning $\true$, and one that computes the probability of evidence (i.e. normalizing constant)
  \item Inductive description has the shape $\te \compiles (\texttt{p}_1, \texttt{p}_2)$. 

  Our goal will be for this compilation to satisfy the following form of semantics-presevation:
  \begin{theorem}[Semantics preservation]
    For well-typed closed term $\te$,
    assume $\te \compiles (\varphi, \varphi_A, w)$. Then, 
    $\dbracket{\te}_D(\true) = {\dbracket{(\varphi \land \varphi_A, w)}} / {\dbracket{(\varphi_A, w)}}$.
  \end{theorem}

  \item Compilation relation:
  \begin{mathpar}
    \inferrule{}{\true \compiles (\true, \true, \emptyset)} \and 
    \inferrule{}{\false \compiles (\false, \true, \emptyset)} \and 
    \inferrule{}{x \compiles (x, \true, \emptyset)} \and 
    \inferrule{\texttt{fresh}~x}
      {\texttt{flip}~\theta \compiles (x, \true, [x \mapsto \theta, \overline{x} \mapsto 1-\theta])} \and
    \inferrule{\te_1 \compiles (\varphi, \varphi_A, w) \and \te_2 \compiles (\varphi', \varphi_A', w')}
      {x \leftarrow \te_1; \te_2 \compiles (\varphi'[\varphi/x], \varphi_A'[\varphi/x] \land \varphi_A, w_1 \cup w_2)} \and
    \inferrule{\te_1 \compiles (\varphi, \true, \emptyset) \and \te_2 \compiles (\varphi', \varphi_A', w)}
      {\texttt{observe}~\te_1; \te_2 \compiles (\varphi', \varphi \land \varphi_A', w)}
  \end{mathpar}

  Most of these rules are unchanged from the previous compilation, except for bind and \texttt{observe}.

  \item Example derivation:
  
  \begin{mathpar}
    \inferrule{\texttt{flip}~\theta \compiles (f, \true, [f \mapsto 1/2, \overline{f} \mapsto 1/2]) \and 
      \inferrule{x \compiles (x, \true, \emptyset) 
      \and 
      \inferrule{x \compiles (x, \true, \emptyset)}{\texttt{return}~x \compiles (x, \true, \emptyset)}}
      {\texttt{observe}~x; \texttt{return}~x \compiles (x, x, \emptyset)}
    }
      {x \leftarrow \texttt{flip}~\theta;~\texttt{observe}~x; \texttt{return}~x \compiles (f, f, [f \mapsto 1/2, \overline{f} \mapsto 1/2])}
  \end{mathpar}

  Check that this satisfies semantics preservation.

\end{itemize}

\section{Some notes on the project}
\begin{itemize}
  \item Instead of compiling to Prop, you can work directly on the BDD
  \item The basic primitive operation you will perform on BDDs is \emph{weighted
  model counting}, which is a more general term for what we've so far calling 
  the semantics of \textsc{Prop}:
  \begin{definition}[Weighted model count]
    Let $\varphi$ be a propositional formula and $w$ be a map from literals
    (assignments to variables) to real-valued weights. The \emph{weighted model count}
    is defined:
    \begin{align}
      \mathtt{WMC}(\varphi,w) \triangleq \sum_{I \models \varphi} \prod_{\ell \in I} w(\ell).
    \end{align}
  \end{definition}
  You can see an example of running a weighted model count in \texttt{disc/lib/kc.ml}
\end{itemize}

% \begin{theorem}
% Assume $\Gamma \vdash \te$ and $\te \compiles \prog$. Then, for any substitution
% $\gamma \in \dbracket{\Gamma}$, is is the case that 
% $\dbracket{\te[\gamma]}_D(\true) = \dbracket{\prog[\gamma]}$.
% \end{theorem}


\section{Sampling \& approximate reasoning}
\begin{itemize}
  \item Up until now we have been exclusively discussing \emph{exact reasoning}:
  computing the exact probability that a program will output a particular value

  \item Problems with exact reasoning:
  \begin{itemize}
    \item State-space explosion
    \item Limited expressive power: how can we handle continuous probability, or loops that may never terminate?
    \item ``All-or-nothing'': exact answer or nothing at all
  \end{itemize}

  \item An alternative is \emph{approximate reasoning}. Many of the most popular PPLs in use 
  today support exclusively this mode of reasoning.\sidenote{For example, Stan~\citep{carpenter2017stan}.}

  \item There is an entirely separate school of PPLs that reason by \emph{sampling}.

  \item The crucial mechanism is the \emph{sample mean}, which gives an estimate of the expectation 
  of a random variable:
  
  \begin{definition}[Expectation]
    Let $(\Omega, \Pr)$ be a probability space and $f : \Omega \rightarrow
    \real$ be a random variable. The \emph{expectation} (or \emph{average value}) of $f$ with respect to 
    $\Pr$ is defined:
    \begin{align}
      \E_{\Pr[f]} \triangleq \sum_{\omega \in \Omega} \Pr(\omega) f(\omega).
    \end{align}
  \end{definition}

  \begin{definition}[Sample mean]
    Let $(\Omega, \Pr)$ be a probability space and $f : \Omega \rightarrow
    \real$ be a random variable. Then, the \emph{sample mean} of $f$ 
    with $N$ samples is defined:
    \begin{align}
      \frac{1}{N} \sum_{\omega_i \sim \Pr}^N f(\omega_i),
    \end{align}
    where the notation $\omega_i \sim \Pr$ denotes drawing a sample $\omega_i$ from 
    the probability distribution $\Pr$.
  \end{definition}

  \marginnote{One may wonder \emph{how quickly} a particular
  estimate of the mean approaches the true value (i.e., how many samples one
  must draw in order to have an accurate estimate with high probability). There
  are many bounds of this sort known broadly as \emph{concentration
  inequalities}; \citet{shalev2014understanding} has a nice summary of some 
  of the useful concentration inequalities that arise in practice in the appendix.}

  \item The reason why we use the sample estimator is that the \emph{law of large numbers} 
  guarantees that, as $N \rightarrow \infty$, the sample mean approaches the 
  expectation, i.e.:
  \begin{align}
    \lim_{N \rightarrow \infty} \frac{1}{N} \sum_{\omega_i \sim \Pr}^N f(\omega_i) = \E_{\Pr}[f].
  \end{align}

  \item What will do is give a semantics to programs in terms of expectations, and then 
  use the expectation estimator in order to get an approximation for the program's 
  behavior

\end{itemize}

\subsection{Sampling semantics for \disc{}}
\marginnote{This secton is based on the semantics given by \citet{culpepper2017contextual}.}
\begin{itemize}
  \item \textbf{Goal}: Give a semantics that draws samples from
  $\dbracket{\te}$, where $\te$ is a (probabilistic) \disc{} term

  \item We still want our semantics to be a \emph{deterministic relation} on terms; how can 
  we draw samples using a deterministic relation?

  \item Solution: Add a \emph{source of randomness} to our context (just like how your 
  computer has \texttt{/dev/rand})

  \item To get our feet wet with this new style of semantics, let's consider a
  tiny sub-language of \disc{} with only the following syntax with only fair
  coin-flips and no observations:
\begin{lstlisting}[mathescape=true]
  e ::= flip 1/2 | x $\leftarrow$ e; e | x | e $\land$ e | e $\lor$ e | $\neg$ e
\end{lstlisting}

  The denotation of this sub-language is inherited from \disc{}. 

  \item To draw samples from this language, we add to our evaluation judgment an
  \emph{infinite stream of fair coin-flips} $\sigma \in \bool^{\nat}$. 
  To make this 
  formal, we will need a way of representing a probability distribution on $\bool^\nat$.
  This is actually much trickier to define than it seems at first! 

  Let's start by defining the probability of any \emph{finite subsequence} of bits:
  \begin{align}
    \Pr((v_1, v_2, \dots, v_n)) = \frac{1}{2^n}
  \end{align}
  
  Clearly as $n \rightarrow \infty$, this probability approaches 0. This seems broken -- 
  how can it be that this defines a valid probability distribution (i.e., sums to 1) on 
  all elements $\sigma \in \bool^\nat$? We will see more on this next time, but for now, 
  accept that this is a valid definition of probability, and we will make this
  more formal later. 

  We will assume that standard properties of probability distributions (such as marginalization)
  hold of $\Pr$, for instance:
  \begin{align}
    \sum_{(v_2, v_3, v_4, \dots) \in \bool^\nat}\Pr(v_1 = \true, v_2, v_3, v_4, \dots) = 1/2
  \end{align}

  \item To handle binding, we will need a way to split this bit-stream up. We notate this 
  $\pi_L$ and $\pi_R$, which split $\sigma$ into 
  two disjoint random sets of bits. We can define these projections in a number of ways, 
  but a simple way is to define $\pi_L$ to take all even bits and $\pi_R$ to take all odd bits:
  \begin{align*}
    \pi_L(v_1, v_2, v_3, \dots) \triangleq (v_2, v_4, v_6, ...) \quad 
    \pi_R(v_1, v_2, v_3, \dots) \triangleq (v_1, v_3, v_5, ...)
  \end{align*}
  
  
  \item Given access to this context, we can define a relation $\sigma \vdash
  \te \Downarrow^S v$ for our tiny language above (showing only the probabilistic terms):
  \begin{mathpar}
    \inferrule{}{v :: \sigma \vdash \texttt{flip} \Downarrow^S v}
    \and
    \inferrule{\pi_L(\sigma) \vdash \te_1 \Downarrow v_1 \and 
    \pi_R(\sigma) \vdash \te_2[v_1/x] \Downarrow v_2
    }{\sigma \vdash x \leftarrow \te_1; \te_2 \Downarrow^S v_2}
  \end{mathpar}

  \item Since our semantics are deterministic, can define a function $ev(\sigma, \te)$ that 
  produces the value that the (closed) term $\te$ evaluates to for $\sigma$:
\end{itemize}


\begin{theorem}
  Let $\Pr$ be the distribution on the infinite bit streams $\bool^\nat$. Assume $\Gamma \vdash \te$.
  Then, for any $\gamma \in \dbracket{\Gamma}$,
  $\E_{\sigma \sim \Pr}[\mathbb{1}(ev(\sigma, \te[\gamma]) = \true)] = \dbracket{\te[\gamma]}(\true)$.
\end{theorem}
\begin{proof}
  The non-probabilistic cases are straightforward. Let's see the \texttt{flip} 
  case first. We can disregard $\gamma$ since this term is closed.
  \begin{align*}
    \E_{\sigma \sim \Pr}[\mathbb{1}(ev(\sigma, \te))] &= 
    \sum_{(v :: \sigma ) \in \bool^\nat} \Pr(v::\sigma) \mathbb{1}[(ev(v::\sigma, \te)) = \true)] \\ 
    &= \Pr(v = \true) \mathbb{1}[\true = \true] + \Pr(v = \false)\mathbb{1}[\false = \true] \\ 
    &= 1/2 = \dbracket{\texttt{flip~1/2}}(\true).
  \end{align*}

  Now for the bind case, $x \leftarrow \te_1; \te_2$. Assume by induction that:
  \begin{itemize}
    \item $\E_{\sigma \sim \Pr}[\mathbb{1}(ev(\sigma, \te_1) = \true)] = \dbracket{\te_1}(\true)$.
    \item $\E_{\sigma \sim \Pr}[\mathbb{1}(ev(\sigma, \te_2) = \true)] = \dbracket{\te_2}(\true)$.
  \end{itemize}
  Proceeding:
  \begin{align*}
    \E_{\sigma \sim \Pr}[\mathbb{1}(ev(x \leftarrow \te_1; \te_2) = \true)]
  \end{align*}
\end{proof}

\begin{itemize}
  \item Now we are left with one final problem: how can we ``run these
  semantics'' on a computer?  I.e., how can we effectively sample from an
  infinite stream of random bits, which seems to require infinite memory (and in
  which each sample has probability 0)? We can \emph{sample the bits lazily}: 
  each time a fresh random bit is required, flip a fair coin to sample it.

  \item Now, let's compare sampling against exact: when might one prefer 
  sampling over exact, and vise versa?
\end{itemize}

\bibliographystyle{plainnat}
\bibliography{../bib}


\end{document}